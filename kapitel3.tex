%%%%%%%%%%%%%%%%%%%%%%%%%%%%%%%%%%%%%%%%%%%%%%%%%%%%%%%%%%%%%%%%%%%%%%%%%%%%%%%%%%%%%%%%%%%%%%%
%% Description:       Programmentwurf advanced software engineering
%% Author:      Manuel Berg, m.berg@enbw.com
%%  -*- coding: utf-8 -*-
%%%%%%%%%%%%%%%%%%%%%%%%%%%%%%%%%%%%%%%%%%%%%%%%%%%%%%%%%%%%%%%%%%%%%%%%%%%%%%%%%%%%%%%%%%%%%%%

\titlespacing*{\chapter}{0pt}{-30mm}{10pt}
\titleformat{\chapter}[display]
  {\normalfont\bfseries}{}{10pt}{\Huge\thechapter.\quad}
  
\chapter{SOLID (8P)}
\pagestyle{scrheadings}
\clearscrheadfoot
\pagenumbering{arabic}
\setcounter{page}{3}
\ofoot[\pagemark]{\pagemark}
%\ohead[\headmark]{\headmark}
\onehalfspacing

\section{Analyse SRP (3P)}
\emph{[Jeweils eine Klasse als positives und negatives Beispiel für SRP; jeweils UML der Klasse und
Beschreibung der Aufgabe bzw. der Aufgaben und möglicher Lösungsweg des Negativ-Beispiels (inkl.
UML)]}

\subsubsection{Positiv-Beispiel}
\subsubsection{Negativ-Beispiel}

\section{Analyse OCP (3P)}
\emph{[Jeweils eine Klasse als positives und negatives Beispiel für OCP; jeweils UML der Klasse und
Analyse mit Begründung, warum das OCP erfüllt/nicht erfüllt wurde – falls erfüllt: warum hier
sinnvoll/welches Problem gab es? Falls nicht erfüllt: wie könnte man es lösen (inkl. UML)?]}

\subsubsection{Positiv-Beispiel}
\subsubsection{Negativ-Beispiel}

\section{Analyse LSP/ISP/DIP (2P)}
\emph{[Jeweils eine Klasse als positives und negatives Beispiel für entweder LSP oder ISP oder DIP); jeweils
UML der Klasse und Begründung, warum man hier das Prinzip erfüllt/nicht erfüllt wird]}
\\
\\
\emph{[Anm.: es darf nur ein Prinzip ausgewählt werden; es darf NICHT z.B. ein positives Beispiel für LSP
und ein negatives Beispiel für ISP genommen werden]}

\subsubsection{Positiv-Beispiel}
\subsubsection{Negativ-Beispiel}