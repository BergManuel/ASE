%%%%%%%%%%%%%%%%%%%%%%%%%%%%%%%%%%%%%%%%%%%%%%%%%%%%%%%%%%%%%%%%%%%%%%%%%%%%%%%
%% Descr:       Sitzungsprotokoll Psych. Grundlagen für Informatiker
%% Author:      Manuel Berg, m.berg@enbw.com
%%  -*- coding: utf-8 -*-
%%%%%%%%%%%%%%%%%%%%%%%%%%%%%%%%%%%%%%%%%%%%%%%%%%%%%%%%%%%%%%%%%%%%%%%%%%%%%%%

\titlespacing*{\chapter}{0pt}{-30mm}{10pt}
\titleformat{\chapter}[display]
  {\normalfont\bfseries}{}{10pt}{\Huge\thechapter.\quad}
  
\chapter{Weitere Prinzipien (8P)}
\pagestyle{scrheadings}
\clearscrheadfoot
\pagenumbering{arabic}
\setcounter{page}{4}
\ofoot[\pagemark]{\pagemark}
%\ohead[\headmark]{\headmark}
\onehalfspacing

\section{Analyse GRASP: Geringe Kopplung (4P)}
\emph{[jeweils eine bis jetzt noch nicht behandelte Klasse als positives und negatives Beispiel geringer
Kopplung; jeweils UML Diagramm mit zusammenspielenden Klassen, Aufgabenbeschreibung der
Klasse und Begründung warum hier eine geringe Kopplung vorliegt bzw. Beschreibung, wie die
Kopplung aufgelöst werden kann]}

\subsubsection{Positiv-Beispiel}
\subsubsection{Negativ-Beispiel}

\section{Analyse GRASP: Hohe Kohäsion (2P)}
\emph{[eine Klasse als positives Beispiel hoher Kohäsion; UML Diagramm und Begründung, warum die
Kohäsion hoch ist]}

\section{DRY (2P)}
\emph{[ein Commit angeben, bei dem duplizierter Code/duplizierte Logik aufgelöst wurde; Code-Beispiele
(vorher/nachher); begründen und Auswirkung beschreiben]}