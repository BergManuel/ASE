%%%%%%%%%%%%%%%%%%%%%%%%%%%%%%%%%%%%%%%%%%%%%%%%%%%%%%%%%%%%%%%%%%%%%%%%%%%%%%%%%%%%%%%%%%%%%%%
%% Description:       Programmentwurf advanced software engineering
%% Author:      Manuel Berg, m.berg@enbw.com
%%  -*- coding: utf-8 -*-
%%%%%%%%%%%%%%%%%%%%%%%%%%%%%%%%%%%%%%%%%%%%%%%%%%%%%%%%%%%%%%%%%%%%%%%%%%%%%%%%%%%%%%%%%%%%%%%

\titlespacing*{\chapter}{0pt}{-30mm}{10pt}
\titleformat{\chapter}[display]
  {\normalfont\bfseries}{}{10pt}{\Huge\thechapter.\quad}
  
\chapter{Einführung (4P)}
\pagestyle{scrheadings}
\clearscrheadfoot
\pagenumbering{arabic}
\setcounter{page}{1}
\ofoot[\pagemark]{\pagemark}
%\ohead[\headmark]{\headmark}
\onehalfspacing

\section{Übersicht über die Applikation (1P)}
\emph{[Was macht die Applikation? Wie funktioniert sie? Welches Problem löst sie/welchen Zweck hat sie?]}
\\
\\
\noindent Die Applikation dient rein der Unterhaltung und simuliert das bekannte deutsche Gesell\-schaftsspiel \emph{Mensch ärgere Dich nicht} mit dem klassischen Spielfeld für vier Spieler. Hierbei kann -- entsprechend dem originalen Brettspiel -- zwischen zwei, drei oder vier Teilnehmenden gewählt werden. Eine Besonderheit der Applikation besteht darin, dass nicht alle Teilnehmenden \enquote{menschlich} sein müssen, sondern nach Belieben auch ein Algorithmus das Würfeln und die Ausführung der Züge übernehmen kann. ToDo: Funktionsweise

\section{Wie startet man die Applikation? (1P)}
\emph{[Wie startet man die Applikation? Was für Voraussetzungen werden benötigt? Schritt-für-Schritt-Anleitung]}
\\
\\
\noindent ToDo: Start in virtueller Ubuntu-Maschine ausprobieren

\section{Technischer Überblick (2P)}
\emph{[Nennung und Erläuterung der Technologien (z.B. Java, MySQL, ...), jeweils Begründung für den
Einsatz der Technologien]}
\\
\\
\noindent Als Programmiersprache wurde \textbf{\emph{Java}} verwendet, da dies in den Anforderungen bereits definiert wurde und wir mit dieser Sprache bisher die meisten Erfahrungen gesammelt haben. Auch in der Vorlesung \enquote{Programmieren} im ersten und zweiten Semester des Studienganges wurden die Grundlagen objektorientierter Programmiersprachen anhand von Java erläutert, weshalb sich dies auch für den Programmentwurf angeboten hat. Bezüglich der \acs{GUI} ist die Entscheidung auf das Framework \textbf{\emph{Swing}} gefallen. Auch hier haben wir in der genannten Vorlesung und auch privat bereits Erfahrungen sammeln können. Da der Fokus des Programmentwurfs nicht auf der \acs{GUI} liegen sollte, haben wir hier die für uns einfachste Variante ausgewählt. Die selben Argumente haben uns auch zu der Entscheidung für das Abhängigkeitsmanagement mit \textbf{\emph{Maven}} geführt.