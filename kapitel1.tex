%%%%%%%%%%%%%%%%%%%%%%%%%%%%%%%%%%%%%%%%%%%%%%%%%%%%%%%%%%%%%%%%%%%%%%%%%%%%%%%%%%%%%%%%%%%%%%%
%% Description:       Programmentwurf advanced software engineering
%% Author:      Manuel Berg, m.berg@enbw.com
%%  -*- coding: utf-8 -*-
%%%%%%%%%%%%%%%%%%%%%%%%%%%%%%%%%%%%%%%%%%%%%%%%%%%%%%%%%%%%%%%%%%%%%%%%%%%%%%%%%%%%%%%%%%%%%%%

\titlespacing*{\chapter}{0pt}{-30mm}{10pt}
\titleformat{\chapter}[display]
  {\normalfont\bfseries}{}{10pt}{\Huge\thechapter.\quad}
  
\chapter{Einführung (4P)}
\pagestyle{scrheadings}
\clearscrheadfoot
\pagenumbering{arabic}
\setcounter{page}{1}
\ofoot[\pagemark]{\pagemark}
%\ohead[\headmark]{\headmark}
\onehalfspacing

\section{Übersicht über die Applikation (1P)}
\emph{[Was macht die Applikation? Wie funktioniert sie? Welches Problem löst sie/welchen Zweck hat sie?]}
\\
\\
\noindent Die Applikation dient rein der Unterhaltung und simuliert das bekannte deutsche Gesell\-schaftsspiel \emph{Mensch ärgere Dich nicht} mit dem klassischen Spielfeld für vier Spieler. Hierbei kann -- entsprechend dem originalen Brettspiel -- zwischen zwei, drei oder vier Teilnehmenden gewählt werden. Eine Besonderheit der Applikation besteht darin, dass nicht alle Teilnehmenden \enquote{menschlich} sein müssen, sondern nach Belieben auch ein Algorithmus das Würfeln und die Ausführung der Züge übernehmen kann. Weitere Details bezüglich der Funktionsweise der Applikation sind im \hyperref[ch:anleitung]{Kapitel 9} nachzulesen. %ToDo: Funktionsweise; Glossar mit Erklärung Spielbrett, Spielfeld

\section{Wie startet man die Applikation? (1P)}
\emph{[Wie startet man die Applikation? Welche Voraussetzungen werden benötigt? Schritt-für-Schritt-Anleitung]}
\\
\\
\noindent Eine Anleitung befindet sich im \hyperref[ch:anleitung]{Kapitel 9}. Auf der Maschine, die zum Ausführen der Anwendung genutzt werden soll, muss lediglich Java 11 installiert sein. Die \textbf{\textbf{.jar}}-Datei, die im entsprechenden GitHub-Repository unter \url{https://github.com/NadineWeiss/project_swe/blob/master/sweProject.jar} zu finden ist, kann dann ausgeführt werden.

\newpage
\section{Technischer Überblick (2P)}
\emph{[Nennung und Erläuterung der Technologien (z.B. Java, MySQL, ...), jeweils Begründung für den
Einsatz der Technologien]}
\\
\\
\noindent Als Programmiersprache wurde Java verwendet, da dies in den Anforderungen bereits definiert wurde und wir mit dieser Sprache bisher die meisten Erfahrungen gesammelt haben. Auch in der Vorlesung \enquote{Programmieren} im ersten und zweiten Semester des Studienganges wurden die Grundlagen objektorientierter Programmiersprachen anhand von Java erläutert, weshalb sich dies auch für den Programmentwurf angeboten hat. Bezüglich der GUI ist die Entscheidung auf das Framework Swing gefallen. Auch hier haben wir in der genannten Vorlesung und auch privat bereits Erfahrungen sammeln können. Da der Fokus des Programmentwurfs nicht auf der GUI liegen sollte, haben wir hier die für uns einfachste Variante ausgewählt. Die selben Argumente haben uns auch zu der Entscheidung für das Abhängigkeitsmanagement mit Maven geführt.