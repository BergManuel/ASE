%%%%%%%%%%%%%%%%%%%%%%%%%%%%%%%%%%%%%%%%%%%%%%%%%%%%%%%%%%%%%%%%%%%%%%%%%%%%%%%
%% Descr:       Sitzungsprotokoll Psych. Grundlagen für Informatiker
%% Author:      Manuel Berg, m.berg@enbw.com
%%  -*- coding: utf-8 -*-
%%%%%%%%%%%%%%%%%%%%%%%%%%%%%%%%%%%%%%%%%%%%%%%%%%%%%%%%%%%%%%%%%%%%%%%%%%%%%%%

\titlespacing*{\chapter}{0pt}{-30mm}{10pt}
\titleformat{\chapter}[display]
  {\normalfont\bfseries}{}{10pt}{\Huge\thechapter.\quad}
  
\chapter{Clean Architecture (8P)}
\pagestyle{scrheadings}
\clearscrheadfoot
\pagenumbering{arabic}
\setcounter{page}{2}
\ofoot[\pagemark]{\pagemark}
%\ohead[\headmark]{\headmark}
\onehalfspacing

\section{Was ist Clean Architecture? (1P)}
\emph{[Allgemeine Beschreibung der Clean Architecture in eigenen Worten]}

\section{Analyse der Dependency Rule (2P)}
\emph{[1 Klasse, die die Dependency Rule einhält und 1 Klasse, die die Dependency Rule verletzt; jeweils
UML der Klasse und Analyse der Abhängigkeiten in beide Richtungen (d.h., von wem hängt die Klasse
ab und wer hängt von der Klasse ab) in Bezug auf die Dependency Rule]}

\subsubsection{Positiv-Beispiel: Dependency Rule}
\subsubsection{Negativ-Beispiel: Dependency Rule}

\section{Analyse der Schichten (5P)}
\emph{[jeweils 1 Klasse zu 2 unterschiedlichen Schichten der Clean-Architecture: jeweils UML der Klasse
(ggf. auch zusammenspielenden Klassen), Beschreibung der Aufgabe, Einordnung mit Begründung in
die Clean-Architecture]}

\subsubsection{Schicht: [Name]}
\subsubsection{Schicht: [Name]}