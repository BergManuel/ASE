%%%%%%%%%%%%%%%%%%%%%%%%%%%%%%%%%%%%%%%%%%%%%%%%%%%%%%%%%%%%%%%%%%%%%%%%%%%%%%%%%%%%%%%%%%%%%%%
%% Description:       Programmentwurf advanced software engineering
%% Author:      Manuel Berg, m.berg@enbw.com
%%  -*- coding: utf-8 -*-
%%%%%%%%%%%%%%%%%%%%%%%%%%%%%%%%%%%%%%%%%%%%%%%%%%%%%%%%%%%%%%%%%%%%%%%%%%%%%%%%%%%%%%%%%%%%%%%

\titlespacing*{\chapter}{0pt}{-30mm}{10pt}
\titleformat{\chapter}[display]
  {\normalfont\bfseries}{}{10pt}{\Huge\thechapter.\quad}
  
\chapter{Unit Tests (8P)}
\pagestyle{scrheadings}
\clearscrheadfoot
\pagenumbering{arabic}
\setcounter{page}{5}
\ofoot[\pagemark]{\pagemark}
%\ohead[\headmark]{\headmark}
\onehalfspacing

\section{10 Unit Tests (2P)}
\emph{[Nennung von 10 Unit-Tests und Beschreibung, was getestet wird]}

\begin{table}[htbp]
\centering
    \begin{tabular}{|l|l|}
        \hline
        \textbf{Unit Test} & \textbf{Beschreibung} \\ \hline
        ~         & ~            \\ \hline
        ~         & ~            \\ \hline
        ~         & ~            \\ \hline
        ~         & ~            \\ \hline
        ~         & ~            \\ \hline
        ~         & ~            \\ \hline
        ~         & ~            \\ \hline
        ~         & ~            \\ \hline
        ~         & ~            \\ \hline
        ~         & ~            \\
        \hline
    \end{tabular}
    \label{Tab:unit_test_table}
    \caption{Übersicht der 10 Unit Tests}
\end{table}

\section{ATRIP: Automatic (1P)}
\emph{[Begründung/Erläuterung, wie ‘Automatic’ realisiert wurde]}

\section{ATRIP: Thorough (1P)}
\emph{[Code Coverage im Projekt analysieren und begründen]}

\section{ATRIP: Professional (1P)}
\emph{[jeweils 1 positves und negatives Beispiele zu ‘Professional’; jeweils Code-Beispiel, Analyse und
Begründung, was professionell/nicht professionell an den Beispielen ist]}

\section{Fakes und Mocks (1P)}
\emph{[Analyse und Begründung des Einsatzes von 2 Fake/Mock-Objekten; zusätzlich jeweils UML
Diagramm der Klasse]}

\newpage
\titlespacing*{\chapter}{0pt}{-30mm}{10pt}
\titleformat{\chapter}[display]
  {\normalfont\bfseries}{}{10pt}{\Huge\thechapter.\quad}
  
\chapter{Domain Driven Design (8P)}
\pagestyle{scrheadings}
\clearscrheadfoot
\pagenumbering{arabic}
\setcounter{page}{6}
\ofoot[\pagemark]{\pagemark}
%\ohead[\headmark]{\headmark}
\onehalfspacing

\section{Ubiquitous Language (2P)}
\emph{[4 Beispiele für die Ubiquitous Language; jeweils Bezeichung, Bedeutung und kurze Begründung,
warum es zur Ubiquitous Language gehört]}

\begin{table}[htbp]
\centering
    \begin{tabular}{|l|l|l|}
        \hline
        \textbf{Bezeichnung} & \textbf{Bedeutung} & \textbf{Begründung} \\ \hline
        ~         & ~          & ~  \\ \hline
        ~         & ~          & ~  \\ \hline
        ~         & ~          & ~  \\ \hline
        ~         & ~          & ~  \\ 

        \hline
    \end{tabular}
    \label{Tab:ddd_examples}
    \caption{4 Beispiele für die Ubiquitous Language}
\end{table}

\section{Repositories (1,5P)}
\emph{[UML, Beschreibung und Begründung des Einsatzes eines Repositories; falls kein Repository
vorhanden: ausführliche Begründung, warum es keines geben kann/hier nicht sinnvoll ist]}

\section{Aggregates (1,5P)}
\emph{[UML, Beschreibung und Begründung des Einsatzes eines Aggregates; falls kein Aggregate
vorhanden: ausführliche Begründung, warum es keines geben kann/hier nicht sinnvoll ist]}

\section{Entities (1,5P)}
\emph{[UML, Beschreibung und Begründung des Einsatzes einer Entity; falls keine Entity vorhanden:
ausführliche Begründung, warum es keines geben kann/hier nicht sinnvoll ist]}

\section{Value Objects (1,5P)}
\emph{[UML, Beschreibung und Begründung des Einsatzes eines Value Objects; falls kein Value Object
vorhanden: ausführliche Begründung, warum es keines geben kann/hier nicht sinnvoll ist]}

\newpage
\titlespacing*{\chapter}{0pt}{-30mm}{10pt}
\titleformat{\chapter}[display]
  {\normalfont\bfseries}{}{10pt}{\Huge\thechapter.\quad}
  
\chapter{Refactoring (8P)}
\pagestyle{scrheadings}
\clearscrheadfoot
\pagenumbering{arabic}
\setcounter{page}{7}
\ofoot[\pagemark]{\pagemark}
%\ohead[\headmark]{\headmark}
\onehalfspacing

\section{Code Smells (2P)}
\emph{[jeweils 1 Code-Beispiel zu 2 unterschiedlichen Code Smells aus der Vorlesung; jeweils Code-Beispiel
und einen möglichen Lösungsweg bzw. den genommen Lösungsweg beschreiben (inkl. (Pseudo-)Code)]}

\section{2 Refactorings (6P)}
\emph{[2 unterschiedliche Refactorings aus der Vorlesung anwenden, begründen, sowie UML vorher/nachher
liefern; jeweils auf die Commits verweisen]}

\newpage
\titlespacing*{\chapter}{0pt}{-30mm}{10pt}
\titleformat{\chapter}[display]
  {\normalfont\bfseries}{}{10pt}{\Huge\thechapter.\quad}
  
\chapter{Entwurfsmuster (8P)}
\pagestyle{scrheadings}
\clearscrheadfoot
\pagenumbering{arabic}
\setcounter{page}{8}
\ofoot[\pagemark]{\pagemark}
%\ohead[\headmark]{\headmark}
\onehalfspacing

\emph{[2 unterschiedliche Entwurfsmuster aus der Vorlesung (oder nach Absprache auch andere) jeweils
sinnvoll einsetzen, begründen und UML-Diagramm]}

\subsubsection{Entwurfsmuster: [Name] (4P)}
\subsubsection{Entwurfsmuster: [Name] (4P)}