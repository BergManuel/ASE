%%%%%%%%%%%%%%%%%%%%%%%%%%%%%%%%%%%%%%%%%%%%%%%%%%%%%%%%%%%%%%%%%%%%%%%%%%%%%%%%%%%%%%%%%%%%%%%
%% Description:       Programmentwurf advanced software engineering
%% Author:      Manuel Berg, m.berg@enbw.com
%%  -*- coding: utf-8 -*-
%%%%%%%%%%%%%%%%%%%%%%%%%%%%%%%%%%%%%%%%%%%%%%%%%%%%%%%%%%%%%%%%%%%%%%%%%%%%%%%%%%%%%%%%%%%%%%%

\documentclass[
   ngerman          % neue deutsche Rechtschreibung
  ,a4paper          % Papiergrösse
% ,twoside          % Zweiseitiger Druck (rechts/links)
% ,10pt             % Schriftgrösse
%  ,11pt
  ,12pt
  ,pdftex
%  ,disable         % Todo-Markierungen auschalten
]{report}

% Codierung der Dateien auswählen:
% \usepackage[latin1]{inputenc}    % Für UNIX mit ISO-LATIN-codierten Dateien
% \usepackage[applemac]{inputenc}  % Für Apple MacOS
% \usepackage[ansinew]{inputenc}   % Für Microsoft Windows
\usepackage[utf8]{inputenc}        % UTF-8 codierte Dateien, Unix

\usepackage{bericht}
\usepackage[document]{ragged2e}
\usepackage{scrlayer-scrpage}
\usepackage{titlesec}
\usepackage{setspace}
\usepackage[backend=bibtex,style=numeric-comp,sorting=none]{biblatex}
\usepackage{makecell}
\usepackage{diagbox}
\setlength\bibitemsep{1.5\itemsep}
\addbibresource{bericht.bib}

%% ACHTUNG, wenn man eine eigene Formatdatei (bericht.fmt) benutzt, werden Änderungen an bericht.sty
%% erst wirksam, wenn die Format-Datei neu erzeugt wurde!
%% (Genauer: Alle Änderungen, die textuell vor der nächsten Zeile ".... endofdump...." stehen
%% werden erst wirksam, wenn die Formatdatei neu erzeugt wurde)
\csname endofdump\endcsname

%%%%%%%%%%%%%%%%%%%%%%%%%%%%%%%%%%%%%%%%%%%%%%%%%%%%%%%%%%%%%%%%%%%%%%%%%%%%%%%
%% Angaben zur Arbeit
%%%%%%%%%%%%%%%%%%%%%%%%%%%%%%%%%%%%%%%%%%%%%%%%%%%%%%%%%%%%%%%%%%%%%%%%%%%%%%%

\newcommand{\Autor}{Nadine Weiß und Manuel Berg}
\newcommand{\MatrikelNummer}{3196898 (Weiß), 5931590 (Berg)}
\newcommand{\Kursbezeichnung}{TINF19B5}

\newcommand{\FirmenName}{EnBW Energie Baden-Württemberg AG}
\newcommand{\FirmenStadt}{Durlacher Allee 93, 76131 Karlsruhe}
\newcommand{\FirmenLogoDeckblatt}{\includegraphics[width=6cm]{enbw}}

\newcommand{\BetreuerFirma}{Andreas Adler, Roman Walz}
%\newcommand{\BetreuerDHBW}{Titel Vorname Nachname}

%%%%%%%%%%%%%%%%%%%%%%%%%%%%%%%%%%%%%%%%%%%%%%%%%%%%%%%%%%%%%%%%%%%%%%%%%%%%%%%%%%%%%

% Wird auf dem Deckblatt und in der Erklärung benutzt:
\newcommand{\Was}{Programmentwurf}
%\newcommand{\Was}{Studienarbeit}
%\newcommand{\Was}{Bachleorarbeit}

%%%%%%%%%%%%%%%%%%%%%%%%%%%%%%%%%%%%%%%%%%%%%%%%%%%%%%%%%%%%%%%%%%%%%%%%%%%%%%%%%%%%%

\newcommand{\Titel}{\enquote{Mensch ärgere Dich nicht}}

\newcommand{\AbgabeDatum}{31.\,05.\,2022}

%\newcommand{\Dauer}{12 Wochen}

%\newcommand{\Abschluss}{Bachelor of Science}

\newcommand{\Studiengang}{Informatik}

\newcommand{\Studiengangsleiter}{Maurice Müller}

\newcommand{\uproman}[1]{\uppercase\expandafter{\romannumeral#1}}
% Befehl für grosse römische Zahlen

\newcommand{\abkspace}{\hspace{1.2cm}}

\newcommand{\myemph}[1]{\textbf{\texttt{#1}}}

\definecolor{light-gray}{gray}{0.85}
%\definecolor{eclipseBlue}{RGB}{42,0.0,255}
%\definecolor{eclipseGreen}{RGB}{63,127,95}
%\definecolor{eclipsePurple}{RGB}{127,0,85}

\hypersetup{%%
  pdfauthor={Manuel Berg},
  pdftitle={Advanced Software Engineering},
  pdfsubject={Advanced Software Engineering}
}

%\setcounter{secnumdepth}{3}

%%%%%%%%%%%%%%%%%%%%%%%%%%%%%%%%%%%%%%%%%%%%%%%%%%%%%%%%%%%%%%%%%%%%%%%%%%%%%%%

% Wenn \includeonly{..} benutzt wird, werden nur diese Kaptitel ausgegeben.
\includeonly{
  kapitel1
 ,kapitel2
 ,kapitel3
 ,kapitel4
 ,kapitel5
}

%%%%%%%%%%%%%%%%%%%%%%%%%%%%%%%%%%%%%%%%%%%%%%%%%%%%%%%%%%%%%%%%%%%%%%%%%%%%%%%

% Benutzt man das "biblatex"-Paket, dann muss das hier stehen:
% siehe auch die mit BIBLATEX markierten Zeilen in bericht.sty
%\addbibresource{bericht.bib}

\begin{document}

%%%%%%%%%%%%%%%%%%%%%%%%%%%%%%%%%%%%%%%%%%%%%%%%%%%%%%%%%%%%%%%%%%%%%%%%%%%%%%%

\begin{titlepage}
\begin{center}
\vspace*{-2cm}
\hfill\includegraphics[width=4.5cm]{dhbw-logo}\\[2cm]
{\Huge \Titel}\\[1.5cm]
{\Huge\scshape \Was}\\[1.5cm]
{\large in der Vorlesung \glqq Advanced Software Engineering\grqq}\\[0.5cm]
{\large im fünften und sechsten Semester}\\[0.5cm]
{\large des Studienganges \Studiengang}\\[0.5cm]
{\large an der}\\[0.5cm]
{\large Dualen Hochschule Baden-Württemberg Karlsruhe}\\[0.5cm]
{\large von}\\[0.5cm]
{\large\bfseries \Autor}\\[1cm]
{\large \AbgabeDatum}
\vfill
\end{center}
\begin{tabular}{l@{\hspace{2cm}}l}
%Bearbeitungszeitraum	         & \Dauer 			\\
Matrikelnummern	                 & \MatrikelNummer		\\
Kurs			         & \Kursbezeichnung		\\
%Ausbildungsbetrieb	         & \FirmenName			\\
			         %& \FirmenStadt			\\
%Betreuer des Ausbildungsbetriebes	 & \BetreuerFirma		\\
%Gutachter der dualen Hochschule	 & \BetreuerDHBW		\\
Dozent	 & \Studiengangsleiter		\\
\end{tabular}
\end{titlepage}

%%%%%%%%%%%%%%%%%%%%%%%%%%%%%%%%%%%%%%%%%%%%%%%%%%%%%%%%%%%%%%%%%%%%%%%%%%%%%%%

%%%%%%%%%%%%%%%%%%%%%%%%%%%%%%%%%%%%%%%%%%%%%%%%%%%%%%%%%%%%%%%%%%%%%%%%%%%%%%%%%%%%%%%%%%%%%%%
%% Description:       Programmentwurf advanced software engineering
%% Author:      Manuel Berg, m.berg@enbw.com
%%  -*- coding: utf-8 -*-
%%%%%%%%%%%%%%%%%%%%%%%%%%%%%%%%%%%%%%%%%%%%%%%%%%%%%%%%%%%%%%%%%%%%%%%%%%%%%%%%%%%%%%%%%%%%%%%

\newpage
\
\onehalfspacing
\pagestyle{scrheadings}
\clearscrheadfoot
\pagenumbering{Roman}
\setcounter{page}{2}
\ofoot[\pagemark]{\pagemark}
%\thispagestyle{empty}
\vspace{-0.7cm}
\justify 
\begin{framed}
\begin{center}
\Large\bfseries Eidesstattliche Erklärung
\end{center}
%\medskip
\noindent
\begin{center}
(gemäß §5(3) der \enquote{Studien- und Prüfungsordnung DHBW Technik} vom 29.\,09.\,2017)\\
\end{center}
Ich versichere hiermit, dass ich mein \Was\ vom 31.\,05.\,2022 mit dem Thema
\enquote{Mensch ärgere Dich nicht}
selbstständig verfasst und keine anderen als die angegebenen Quellen und Hilfsmittel benutzt habe.
\vspace{0.5cm}
%\noindent
%\underline{\hspace{4cm}}\hfill\underline{\hspace{6cm}}\\
%Ort~~~~~Datum\hfill Unterschrift\hspace{4cm}

%% Ort und Datum  
%%\vspace{1,5 cm} 
%\begin{tabular}{p{7cm}p{.5cm}l}
%\dotfill \\ 
%Ort, Datum
%\end{tabular}% 
%
%% Hier kommen die Unterschriten hin
%%\vspace{1,5 cm} 
%\begin{tabular}{p{7cm}p{.5cm}l}
%\dotfill \\ 
%Unterschrift 
%\end{tabular}%
\begin{center}
\hspace*{\fill}\begin{tabular}{@{}l@{}}\hline
\makebox[15.3cm]{Ort, Datum \hspace{7cm} Unterschrift}
\end{tabular}
\end{center}




\end{framed}

\newpage
\onehalfspacing

%\vspace{-3cm}
%\begin{framed}
%\begin{center}
%\Large\bfseries Sperrvermerk
%\end{center}
%\medskip
%\noindent
%Die  vorliegende  Projektarbeit  beinhaltet  interne  und  vertrauliche  Informationen  der EnBW  Energie Baden-Württemberg  AG, der  EnBW  Kernkraft  GmbH, sowie der Gesellschaft für nukleares Reststoffrecycling mbH.  Die Weitergabe des Inhalts dieser Arbeit, beiliegender Zeichnungen oder anderer Daten im  Gesamten  oder  in  Teilen  ist  grundsätzlich  untersagt.  Es  dürfen  keinerlei  Kopien oder Abschriften -- in schriftlicher oder digitaler Form -- angefertigt oder verteilt werden. Ausnahmen  bedürfen  der  schriftlichen  Genehmigung  der beteiligten Unternehmen.  Die  Arbeit  ist  nur  den  Korrektoren  sowie  den  Mitgliedern  des Prüfungsausschusses zugänglich zu machen.
%\end{framed}

\vspace{9cm}
\newcommand{\latex}{\LaTeX\xspace}
\begin{framed}
\justify \textbf{Hinweis:} Der \latex-Quellcode dieses Dokumentes basiert auf der \href{https://www.karlsruhe.dhbw.de/inf/studienverlauf-organisatorisches.html}{\enquote{Vorlage für Berichte der DHBW Karlsruhe}}, welche freundlicherweise von \href{mailto:juergen.vollmer@dhbw-karlsruhe.de}{Prof.\,Dr.\:Jürgen Vollmer} zur Verfügung gestellt wurde.
%\end{flushleft}
\end{framed}


%%%%%%%%%%%%%%%%%%%%%%%%%%%%%%%%%%%%%%%%%%%%%%%%%%%%%%%%%%%%%%%%%%%%%%%%%%%%%%%
\endinput
%%%%%%%%%%%%%%%%%%%%%%%%%%%%%%%%%%%%%%%%%%%%%%%%%%%%%%%%%%%%%%%%%%%%%%%%%%%%%%%


%%%%%%%%%%%%%%%%%%%%%%%%%%%%%%%%%%%%%%%%%%%%%%%%%%%%%%%%%%%%%%%%%%%%%%%%%%%%%%%
%\justify
%\vspace*{-6cm}
%\begin{abstract}
%\thispagestyle{plain}
%\setcounter{page}{4}
%Das Thema der Projektarbeit ist die Einführung eines Intranets mit Anbindung an Informationsbildschirme innerhalb der kritischen Infrastruktur der Gesellschaft für nukleares Reststoffrecycling mbH (\acs{GNR}). Um dies zu ermöglichen, ist zunächst eine genaue Analyse der Rahmenbedingungen notwendig. Das grobe Lastenheft, welches zum Beginn der Projektarbeit vorliegt, muss in Absprache mit der Geschäftsführung der \acs{GNR} unter Berücksichtigung technischer Möglichkeiten verfeinert werden. Im Rahmen eines ausführ\-lichen Produktvergleiches sollen die am Markt verfügbaren Optionen aufgezeigt werden, unter denen abgewogen werden muss. Ein zentraler Aspekt der gesamten Projektarbeit ist, dass das Intranet eine einfache und übersichtliche Möglichkeit zur redaktionellen Pflege der Inhalte bieten soll. Falls sich bei dem gewählten Produkt mehrere mögliche Implementierungsvarianten anbieten, so soll eine Empfehlung unter Berücksichtigung des besonderen Anwendungsfalls bei der \acs{GNR} ausgesprochen werden. Für die Informationsbildschirme an den Standorten der \acs{GNR} ist gemäß dem Lastenheft eine gesonderte Intranet-Seite aufzubauen. Das gesamte Projekt ist weitestmöglich in den Produktivbetrieb zu überführen, sofern es die informationstechnischen Sicherheitsprüfungen und der eventuelle Einsatz von Fremdpersonal zeitlich zulassen.\\
%
%\vspace{0.5cm}
%\begin{center}
%\textbf{Abstract}
%\end{center}
%The topic of this project is the introduction of an intranet with connection to information monitors within the critical infrastrucure given at \acs{GNR}. In order to achieve this, it is necessary to thoroughly analyse the general conditions of the present environment. The vague requirement specification which exists upon the start of this project has to be refined in consultation with the executive board of \acs{GNR}, keeping the technical possibilities in mind. An extensive comparison of the available products will be conducted. One key aspect of the whole project is that the intranet should offer an easy and comprehensible way to allow editorial maintenance. Should the chosen product offer multiple variants of implementation, a decision should be made considering the special use case at \acs{GNR}. According to the requirement specification, a separate intranet page for the information monitors will be built. The whole project is to be implemented as far as the IT security guidelines and the possible hiring of external staff allow it.
%
%\end{abstract}

\newpage
\pagestyle{scrheadings}
\clearscrheadfoot
\pagenumbering{Roman}
\setcounter{page}{3}
\ofoot[\pagemark]{\pagemark}
\pdfbookmark[section]{\contentsname}{toc}
\tableofcontents           % Inhaltsverzeichnis hier ausgeben
\listoffigures             % Liste der Abbildungen
\addcontentsline{toc}{chapter}{\listfigurename}
\listoftables              % Liste der Tabellen
\addcontentsline{toc}{chapter}{\listtablename}
%\lstlistoflistings         % Liste der Listings
%\listofequations           % Liste der Formeln

% Jetzt kommt der "eigentliche" Text
%%%%%%%%%%%%%%%%%%%%%%%%%%%%%%%%%%%%%%%%%%%%%%%%%%%%%%%%%%%%%%%%%%%%%%%%%%%%%%%%
%% Descr:       Sitzungsprotokoll Psych. Grundlagen für Informatiker
%% Author:      Manuel Berg, m.berg@enbw.com
%%  -*- coding: utf-8 -*-
%%%%%%%%%%%%%%%%%%%%%%%%%%%%%%%%%%%%%%%%%%%%%%%%%%%%%%%%%%%%%%%%%%%%%%%%%%%%%%%

\chapter*{Abkürzungsverzeichnis}                   % chapter*{..} -->   keine Nummer, kein "Kapitel"
						         				   % Nicht ins Inhaltsverzeichnis
						         				   
\addcontentsline{toc}{chapter}{Akürzungsverzeichnis}   % Damit das doch ins Inhaltsverzeichnis kommt

% Hier werden die Abkürzungen definiert
\begin{acronym}[DHBW]

% \acro{Name}{Darstellung der Abkürzung}{Langform der Abkürzung}

\acro{BSI}[BSI]{\abkspace Bundesamt für Sicherheit in der Informationstechnik}
\acro{GI}[GI]{\abkspace Gesellschaft für Informatik e.V.}

 % Wenn nicht benutzt, erscheint diese Abk. nicht in der Liste
 %\acro{NUA}{Not Used Acronym}
 
\end{acronym}
              % Abkürzungsverzeichnis
%%%%%%%%%%%%%%%%%%%%%%%%%%%%%%%%%%%%%%%%%%%%%%%%%%%%%%%%%%%%%%%%%%%%%%%%%%%%%%%
%% Descr:       Sitzungsprotokoll Psych. Grundlagen für Informatiker
%% Author:      Manuel Berg, m.berg@enbw.com
%%  -*- coding: utf-8 -*-
%%%%%%%%%%%%%%%%%%%%%%%%%%%%%%%%%%%%%%%%%%%%%%%%%%%%%%%%%%%%%%%%%%%%%%%%%%%%%%%

\titlespacing*{\chapter}{0pt}{-30mm}{10pt}
\titleformat{\chapter}[display]
  {\normalfont\bfseries}{}{10pt}{\Huge\thechapter.\quad}
  
\chapter{Einführung (4P)}
\pagestyle{scrheadings}
\clearscrheadfoot
\pagenumbering{arabic}
\setcounter{page}{1}
\ofoot[\pagemark]{\pagemark}
%\ohead[\headmark]{\headmark}
\onehalfspacing

\section{Übersicht über die Applikation (1P)}
\emph{[Was macht die Applikation? Wie funktioniert sie? Welches Problem löst sie/welchen Zweck hat sie?]}

\section{Wie startet man die Applikation? (1P)}
\emph{[Wie startet man die Applikation? Was für Voraussetzungen werden benötigt? Schritt-für-Schritt-
Anleitung]}

\section{Technischer Überblick (2P)}
\emph{[Nennung und Erläuterung der Technologien (z.B. Java, MySQL, ...), jeweils Begründung für den
Einsatz der Technologien]}
%%%%%%%%%%%%%%%%%%%%%%%%%%%%%%%%%%%%%%%%%%%%%%%%%%%%%%%%%%%%%%%%%%%%%%%%%%%%%%%%%%%%%%%%%%%%%%%
%% Description:       Programmentwurf advanced software engineering
%% Author:      Manuel Berg, m.berg@enbw.com
%%  -*- coding: utf-8 -*-
%%%%%%%%%%%%%%%%%%%%%%%%%%%%%%%%%%%%%%%%%%%%%%%%%%%%%%%%%%%%%%%%%%%%%%%%%%%%%%%%%%%%%%%%%%%%%%%

\titlespacing*{\chapter}{0pt}{-30mm}{10pt}
\titleformat{\chapter}[display]
  {\normalfont\bfseries}{}{10pt}{\Huge\thechapter.\quad}
  
\chapter{Clean Architecture (8P)}
\pagestyle{scrheadings}
\clearscrheadfoot
\pagenumbering{arabic}
\setcounter{page}{2}
\ofoot[\pagemark]{\pagemark}
%\ohead[\headmark]{\headmark}
\onehalfspacing

\section{Was ist Clean Architecture? (1P)}
\emph{[Allgemeine Beschreibung der Clean Architecture in eigenen Worten]}
\\
\\
\noindent Unter der \emph{Clean Architecture} versteht man eine Architekturrichtlinie. Die Grundidee kann man sich als eine Zwiebel vorstellen, welche aus fünf Schichten besteht. Die fünf Schichten Plugins, Adapters, Application Code, Domain Code sowie Abstraction Code stellen hierbei die Bereiche der Software dar. So bildet der Abstraction Code den innersten Kern, und die Plugins bilden die äußerste Hülle. Je weiter ein Ring vom Kern entfernt ist, desto \enquote{sichtbarer} ist er in der tatsächlichen Anwendung und desto mehr Änderungen erfährt er.

Ein zentraler Aspekt der \emph{Clean Architecture} ist hierbei die sogenannte \emph{Dependency Rule}. Diese besagt, dass Abhängigkeiten in dem zuvor erwähnten Modell nur nach Innen -- also in Richtung des Kerns (Abstraction Code) -- zeigen dürfen. So darf die Adapters-Schicht beispielsweise von der Schicht des Application Codes abhängig sein, jedoch nicht von den Plugins.

\section{Analyse der Dependency Rule (2P)}
\emph{[1 Klasse, die die Dependency Rule einhält und 1 Klasse, die die Dependency Rule verletzt; jeweils
UML der Klasse und Analyse der Abhängigkeiten in beide Richtungen (d.h., von wem hängt die Klasse
ab und wer hängt von der Klasse ab) in Bezug auf die Dependency Rule]}

\subsubsection{Positiv-Beispiel: Dependency Rule}
\noindent Die Klasse \emph{Graph} befindet sich in der Adapters-Schicht und realisiert die Knoten und Kanten des Spielfeldes. Da dies nur die Struktur des Spielfeldes enthält und in keinster Weise mit Spielständen zu tun hat, sind keine Abhängigkeiten in Richtung äußerer Schichten enthalten.

\newpage
\noindent
\begin{figure}[htbp]
\centering
\centerline{\includegraphics[scale=.5]{dependencyrule_klasse_graph}}
\caption{Positiv-Beispiel Dependency Rule [Eigene Darstellung aus \emph{IntelliJ}]}
\label{fig:dependencyrulepositiv}
\end{figure}
\\
\noindent In der \hyperref[fig:dependencyrulepositiv]{Abbildung 2.1} sind alle sichtbaren Klassen in derselben Schicht wie die \emph{Graph}-Klasse. Die in den Packages enthaltenen Klassen wiederum befinden sich in anderen Schichten. Da es von der \emph{Graph}-Klasse keine Abhängigkeit auf eines der anderen Packages (und somit auch nicht auf andere Schichten) gibt, ist dies ein Positivbeispiel für die Einhaltung der Dependency Rule.

\newpage
\noindent

\subsubsection{Negativ-Beispiel: Dependency Rule}
\noindent Bei der Klasse \emph{GameService} handelt es sich um Application Code, denn die Klasse kümmert sich um einen reibungslosen Spielablauf. Deshalb sollten hier auch grundsätzlich keine Abhängigkeiten auf die \acs{GUI} vorhanden sein.

\begin{figure}[htbp]
\centering
\centerline{\includegraphics[scale=.5]{dependencyrule_klasse_gameservice}}
\caption{Negativ-Beispiel Dependency Rule [Eigene Darstellung aus \emph{IntelliJ}]}
\label{fig:dependencyrulenegativ}
\end{figure}

\noindent In \hyperref[fig:dependencyrulepositiv]{Abbildung 2.2} ist zu erkennen, dass die \emph{GameService}-Klasse Abhängigkeiten in Richtung der Klassen des Packages \emph{visualization} aufweist. Die Klassen dieses Packages sind verantwortlich für die \acs{GUI} und gehören somit zu äußersten Schicht. Aus diesem Grund liegt hier ein Negativ-Beispiel vor.

\newpage

\section{Analyse der Schichten (5P)}
\emph{[jeweils 1 Klasse zu 2 unterschiedlichen Schichten der Clean-Architecture: jeweils UML der Klasse
(ggf. auch zusammenspielenden Klassen), Beschreibung der Aufgabe, Einordnung mit Begründung in
die Clean-Architecture]}

\subsubsection{Schicht: Adapters}
\begin{figure}[htbp]
\centering
\centerline{\includegraphics[scale=.35]{node}}
\caption{Klasse der Adapters-Schicht [Eigene Darstellung aus \emph{IntelliJ}]}
\label{fig:dependencyrulenegativ}
\end{figure}
\noindent Sie Klasse \emph{Node} entspricht einem Spielfeld bei \emph{Mensch ärgere Dich nicht}. Jedes Feld, welches von einer Spielfigur besetzt werden kann, wird durch diese Klasse einem Feldtypen zugeordnet. 

\subsubsection{Schicht: [Name]}
%%%%%%%%%%%%%%%%%%%%%%%%%%%%%%%%%%%%%%%%%%%%%%%%%%%%%%%%%%%%%%%%%%%%%%%%%%%%%%%
%% Descr:       Sitzungsprotokoll Psych. Grundlagen für Informatiker
%% Author:      Manuel Berg, m.berg@enbw.com
%%  -*- coding: utf-8 -*-
%%%%%%%%%%%%%%%%%%%%%%%%%%%%%%%%%%%%%%%%%%%%%%%%%%%%%%%%%%%%%%%%%%%%%%%%%%%%%%%

\titlespacing*{\chapter}{0pt}{-30mm}{10pt}
\titleformat{\chapter}[display]
  {\normalfont\bfseries}{}{10pt}{\Huge\thechapter.\quad}
  
\chapter{SOLID (8P)}
\pagestyle{scrheadings}
\clearscrheadfoot
\pagenumbering{arabic}
\setcounter{page}{3}
\ofoot[\pagemark]{\pagemark}
%\ohead[\headmark]{\headmark}
\onehalfspacing

\section{Analyse SRP (3P)}
\emph{[Jeweils eine Klasse als positives und negatives Beispiel für SRP; jeweils UML der Klasse und
Beschreibung der Aufgabe bzw. der Aufgaben und möglicher Lösungsweg des Negativ-Beispiels (inkl.
UML)]}

\subsubsection{Positiv-Beispiel}
\subsubsection{Negativ-Beispiel}

\section{Analyse OCP (3P)}
\emph{[Jeweils eine Klasse als positives und negatives Beispiel für OCP; jeweils UML der Klasse und
Analyse mit Begründung, warum das OCP erfüllt/nicht erfüllt wurde – falls erfüllt: warum hier
sinnvoll/welches Problem gab es? Falls nicht erfüllt: wie könnte man es lösen (inkl. UML)?]}

\subsubsection{Positiv-Beispiel}
\subsubsection{Negativ-Beispiel}

\section{Analyse LSP/ISP/DIP (2P)}
\emph{[Jeweils eine Klasse als positives und negatives Beispiel für entweder LSP oder ISP oder DIP); jeweils
UML der Klasse und Begründung, warum man hier das Prinzip erfüllt/nicht erfüllt wird]}
\\
\\
\emph{[Anm.: es darf nur ein Prinzip ausgewählt werden; es darf NICHT z.B. ein positives Beispiel für LSP
und ein negatives Beispiel für ISP genommen werden]}

\subsubsection{Positiv-Beispiel}
\subsubsection{Negativ-Beispiel}
%%%%%%%%%%%%%%%%%%%%%%%%%%%%%%%%%%%%%%%%%%%%%%%%%%%%%%%%%%%%%%%%%%%%%%%%%%%%%%%%%%%%%%%%%%%%%%%
%% Description:       Programmentwurf advanced software engineering
%% Author:      Manuel Berg, m.berg@enbw.com
%%  -*- coding: utf-8 -*-
%%%%%%%%%%%%%%%%%%%%%%%%%%%%%%%%%%%%%%%%%%%%%%%%%%%%%%%%%%%%%%%%%%%%%%%%%%%%%%%%%%%%%%%%%%%%%%%

\titlespacing*{\chapter}{0pt}{-30mm}{10pt}
\titleformat{\chapter}[display]
  {\normalfont\bfseries}{}{10pt}{\Huge\thechapter.\quad}
  
\chapter{Weitere Prinzipien (8P)}
\pagestyle{scrheadings}
\clearscrheadfoot
\pagenumbering{arabic}
\setcounter{page}{15}
\ofoot[\pagemark]{\pagemark}
%\ohead[\headmark]{\headmark}
\onehalfspacing

\section{Analyse GRASP: Geringe Kopplung (4P)}
\emph{[jeweils eine bis jetzt noch nicht behandelte Klasse als positives und negatives Beispiel geringer
Kopplung; jeweils UML Diagramm mit zusammenspielenden Klassen, Aufgabenbeschreibung der
Klasse und Begründung warum hier eine geringe Kopplung vorliegt bzw. Beschreibung, wie die
Kopplung aufgelöst werden kann]}

\subsubsection{Positiv-Beispiel}
\noindent Eine geringe Kopplung wurde durch den Einsatz des Observer-Patterns zwischen der für die Visualisierung zuständigen \enquote{GameFrame}-Klasse und der für den Spielablauf zuständigen \enquote{GameService}- Klasse erreicht. Das heißt, wenn in der UI durch den Benutzer eine Aktion ausgeführt wird, wird die \enquote{GameService}-Klasse benachrichtigt. Bei der Benachrichtigung weiß das benachrichtigende Objekt nichts Näheres über das zu benachrichtigende Objekt. Anders ausgedrückt, das Observable kennt nur die Observer-Schnittstelle. Dadurch wurde hier eine lose Kopplung geschaffen. 

\begin{figure}[htbp]
\centering
\centerline{\includegraphics[scale=.45]{grasp1}}
\caption{Positiv-Beispiel GRASP [Eigene Darstellung aus \emph{IntelliJ}]}
\label{fig:grasp1}
\end{figure}

\newpage
\subsubsection{Negativ-Beispiel}
\noindent Eine starke Kopplung liegt zwischen der \enquote{Algorithm}- Klasse und der \enquote{AbstractControl\-Mechnism}-Klasse vor. Erstere berechnet einen Zug, der nicht durch einen Benutzer ausgeführt wird -- wenn also ein Benutzer gegen den Algorithmus spielt. Die \enquote{AbstractControl\-Mechnism}-Klasse ruft die Zugberechnung auf. Die Klasse ist abstrakt, da der Aufruf für ein 4-Personen- oder ein 6-Personen-Spielfeld gleichermaßen gilt. 

Eine starke Kopplung liegt vor, da hier auf die konkrete \enquote{Algorithm}- Klasse zuge\-griffen wird. Die Kopplung könnte man durch die Implementierung einer Algorithm-Schnittstelle lockern. Dadurch könnten auch verschiedene Algorithmen implementiert werden.

\begin{figure}[htbp]
\centering
\centerline{\includegraphics[scale=.6]{grasp2}}
\caption{Negativ-Beispiel GRASP [Eigene Darstellung aus \emph{IntelliJ}]}
\label{fig:grasp2}
\end{figure}

\newpage
\section{Analyse GRASP: Hohe Kohäsion (2P)}
\emph{[eine Klasse als positives Beispiel hoher Kohäsion; UML Diagramm und Begründung, warum die Kohäsion hoch ist]}

\vspace{.4cm}

\noindent Die Kohäsion erhöht sich, je mehr Verantwortlichkeiten und Teilaufgaben in andere Klassen ausgelagert sind. Ein Beispiel für hohe Kohäsion wäre hier der dem Spielfeld unterliegende Graph.

\begin{figure}[htbp]
\centering
\centerline{\includegraphics[scale=.5]{grasp3}}
\caption{GRASP: Hohe Kohäsion [Eigene Darstellung aus \emph{IntelliJ}]}
\label{fig:grasp3}
\end{figure}

\newpage

\noindent In der \enquote{Graph}-Klasse wird der Graph aufgebaut. Hierzu wurden aber nicht alle Verantwortlichkeiten in dieser Klasse belassen, sondern in vier weitere Klassen ausgelagert.

\begin{itemize}
    \item Die \enquote{Node}-Klasse stellt einen Knoten dar und enthält den Feldtyp (Siehe \enquote{FieldType}) dieses Knotens und die ausgehenden Kanten (Siehe \enquote{Edge})
    \item Das \enquote{FieldType}-Enum gibt den Feldtyp an. Das kann ein neutrales Feld sein oder zum Beispiel ein rotes Startfeld oder ein gelbes Zielfeld.
    \item Die \enquote{Edge}-Klasse enthält den Zielknoten, die Richtung (Siehe \enquote{Direction}) und die Information, ob es ein Default-Kante ist. Letzteres bedeutet, dass die Kante von allen Spielfiguren befahren werden kann. Dies ist zum Beispiel bei der Kante zu den Zielfeldern der jeweiligen Farben nicht der Fall.
    \item Das \enquote{Direction}-Enum gibt an, in welche Richtung die Kante geht.
\end{itemize}

\newpage
\section{DRY (2P)}
\emph{[ein Commit angeben, bei dem duplizierter Code/duplizierte Logik aufgelöst wurde; Code-Beispiele
(vorher/nachher); begründen und Auswirkung beschreiben]}

\noindent Die Klasse \enquote{GraphUtilities} wurde hinzugefügt, da die jetzt enthaltenen vier Methoden früher von den drei Klassen \enquote{Graph}, \enquote{AbstractBoard} und \enquote{ControlMechanismFour} jeweils extra implementiert wurden und sich der Code dadurch dupliziert hat. Das Ergebnis:

\begin{figure}[htbp]
\centering
\centerline{\includegraphics[scale=.6]{dry1}}
\caption{DRY [Eigene Darstellung aus \emph{IntelliJ}]}
\label{fig:dry1}
\end{figure}

\noindent In der \enquote{Graph}-Klasse waren früher alle vier Methoden, in der \enquote{AbstractBoard}-Klasse war nur die \enquote{getInitType}-Methode und in der \enquote{ControlMechanismFour}-Klasse waren ebenfalls alle vier Methoden vorhanden. Der Unterschied zum jetzigen Stand besteht darin, dass die Methoden mittlerweile statisch gemacht worden sind. Da sich innerhalb der Methode nichts verändert hat, sei hier der frühere Stand nicht als Codebeispiel aufgezeigt.

Positiv ist, dass sich der Code durch das Eliminieren von Duplikaten reduziert hat. Außerdem ist aus den einzelnen Klassen Code, der nicht zu deren Verantwortlichkeiten gezählt hat, herausgenommen worden. Die Klasse \enquote{GraphUtilities} existierte bis zu dem Commit \textbf{\texttt{75cf8293f43de67a5e087053f9a0e3cc762132a5}} vom 26.05.2022.
%%%%%%%%%%%%%%%%%%%%%%%%%%%%%%%%%%%%%%%%%%%%%%%%%%%%%%%%%%%%%%%%%%%%%%%%%%%%%%%%%%%%%%%%%%%%%%%
%% Description:       Programmentwurf advanced software engineering
%% Author:      Manuel Berg, m.berg@enbw.com
%%  -*- coding: utf-8 -*-
%%%%%%%%%%%%%%%%%%%%%%%%%%%%%%%%%%%%%%%%%%%%%%%%%%%%%%%%%%%%%%%%%%%%%%%%%%%%%%%%%%%%%%%%%%%%%%%

\titlespacing*{\chapter}{0pt}{-30mm}{10pt}
\titleformat{\chapter}[display]
  {\normalfont\bfseries}{}{10pt}{\Huge\thechapter.\quad}
  
\chapter{Unit Tests (8P)}
\pagestyle{scrheadings}
\clearscrheadfoot
\pagenumbering{arabic}
\setcounter{page}{5}
\ofoot[\pagemark]{\pagemark}
%\ohead[\headmark]{\headmark}
\onehalfspacing

\section{10 Unit Tests (2P)}
\emph{[Nennung von 10 Unit-Tests und Beschreibung, was getestet wird]}

\begin{table}[htbp]
\centering
    \begin{tabular}{|l|l|}
        \hline
        \textbf{Unit Test} & \textbf{Beschreibung} \\ \hline
        ~         & ~            \\ \hline
        ~         & ~            \\ \hline
        ~         & ~            \\ \hline
        ~         & ~            \\ \hline
        ~         & ~            \\ \hline
        ~         & ~            \\ \hline
        ~         & ~            \\ \hline
        ~         & ~            \\ \hline
        ~         & ~            \\ \hline
        ~         & ~            \\
        \hline
    \end{tabular}
    \label{Tab:unit_test_table}
    \caption{Übersicht der 10 Unit Tests}
\end{table}

\section{ATRIP: Automatic (1P)}
\emph{[Begründung/Erläuterung, wie ‘Automatic’ realisiert wurde]}

\section{ATRIP: Thorough (1P)}
\emph{[Code Coverage im Projekt analysieren und begründen]}

\section{ATRIP: Professional (1P)}
\emph{[jeweils 1 positves und negatives Beispiele zu ‘Professional’; jeweils Code-Beispiel, Analyse und
Begründung, was professionell/nicht professionell an den Beispielen ist]}

\section{Fakes und Mocks (1P)}
\emph{[Analyse und Begründung des Einsatzes von 2 Fake/Mock-Objekten; zusätzlich jeweils UML
Diagramm der Klasse]}

\newpage
\titlespacing*{\chapter}{0pt}{-30mm}{10pt}
\titleformat{\chapter}[display]
  {\normalfont\bfseries}{}{10pt}{\Huge\thechapter.\quad}
  
\chapter{Domain Driven Design (8P)}
\pagestyle{scrheadings}
\clearscrheadfoot
\pagenumbering{arabic}
\setcounter{page}{6}
\ofoot[\pagemark]{\pagemark}
%\ohead[\headmark]{\headmark}
\onehalfspacing

\section{Ubiquitous Language (2P)}
\emph{[4 Beispiele für die Ubiquitous Language; jeweils Bezeichung, Bedeutung und kurze Begründung,
warum es zur Ubiquitous Language gehört]}

\begin{figure}[htbp]
\centering
\centerline{\includegraphics[scale=.7]{ubiquitous}}
\caption{4 Beispiele für die Ubiquitous Language}
\label{fig:ubiquitous}
\end{figure}

% \begin{table}[htbp]
% \centering
%     \begin{tabular}{|l|l|l|}
%         \hline
%         \textbf{Bezeichnung} & \textbf{Bedeutung} & \textbf{Begründung} \\ \hline
%         ~         & ~          & ~  \\ \hline
%         ~         & ~          & ~  \\ \hline
%         ~         & ~          & ~  \\ \hline
%         ~         & ~          & ~  \\ 

%         \hline
%     \end{tabular}
%     \label{Tab:ddd_examples}
%     \caption{4 Beispiele für die Ubiquitous Language}
% \end{table}

\section{Repositories (1,5P)}
\emph{[UML, Beschreibung und Begründung des Einsatzes eines Repositories; falls kein Repository
vorhanden: ausführliche Begründung, warum es keines geben kann/hier nicht sinnvoll ist]}

\noindent Das Repository wird durch die Klasse \enquote{GameIO} repräsentiert und bietet Zugriff auf persistenten Speicher. Außerdem wird die konkret verwendete Speichertechnologie vor dem Domain Code verborgen. In unserem Beispiel ist es das Abspeichern und Laden von Spielständen, was mithilfe von JSON umgesetzt wird. Dadurch kann ein Spiel unterbrochen und später fortgesetzt werden.

\section{Aggregates (1,5P)}
\emph{[UML, Beschreibung und Begründung des Einsatzes eines Aggregates; falls kein Aggregate
vorhanden: ausführliche Begründung, warum es keines geben kann/hier nicht sinnvoll ist]}

\begin{figure}[htbp]
\centering
\centerline{\includegraphics[scale=.6]{aggregat}}
\caption{Aggregate [Eigene Darstellung aus \emph{IntelliJ}]}
\label{fig:aggregat}
\end{figure}

\noindent Die in \hyperref[fig:aggregat]{Abbildung 6.2} dargestellten Klassen gehören zu einem Aggregat. Dies ist eine gemeinsam verwaltete Einheit und ist innerhalb seiner Grenzen konsistent. Durch diese Konsistenzgrenze war es sinnvoll, das Aggregat für das Spielbrett einzusetzen, da beispielsweise auch die Züge stets konsistent sein müssen. Als Aggregatswurzel fungiert das \enquote{BoardInterface}. Dieses kontrolliert alle Zugriffe auf das Aggregat. Wenn eines der vier oben dargestellten Packages Änderungen am Spielbrett vornehmen möchte, geht dies nur über die Aggregatswurzel. 

\section{Entities (1,5P)}
\emph{[UML, Beschreibung und Begründung des Einsatzes einer Entity; falls keine Entity vorhanden:
ausführliche Begründung, warum es keines geben kann/hier nicht sinnvoll ist]}

\noindent \textbf{ToDo: Player UUID}

\section{Value Objects (1,5P)}
\emph{[UML, Beschreibung und Begründung des Einsatzes eines Value Objects; falls kein Value Object
vorhanden: ausführliche Begründung, warum es keines geben kann/hier nicht sinnvoll ist]}

\begin{figure}[htbp]
\centering
\centerline{\includegraphics[scale=.6]{valueobject}}
\caption{Value Object [Eigene Darstellung aus \emph{IntelliJ}]}
\label{fig:valueobject}
\end{figure}

\noindent Ein Value Object ist hier eine Spielfigur. Diese hat als Eigenschaft nur die Farbe, welche über den gesamten Zeitraum hinweg unveränderlich bleibt. Die Änderung der Farbe ist nur durch die Konstruktion eines neuen Objektes möglich. Die Spielfigur besitzt keine eigene Identität und es ist auch kein Lebenszyklus erkennbar.

Bezüglich des \enquote{GamePiece} sind das Überschreiben von \enquote{equals()} und \enquote{hashCode()} sowie die weiteren Vorgaben aus der Vorlesung zur Implementierung von Value Objects in Java beachtet worden.

\newpage
\titlespacing*{\chapter}{0pt}{-30mm}{10pt}
\titleformat{\chapter}[display]
  {\normalfont\bfseries}{}{10pt}{\Huge\thechapter.\quad}
  
\chapter{Refactoring (8P)}
\pagestyle{scrheadings}
\clearscrheadfoot
\pagenumbering{arabic}
\setcounter{page}{7}
\ofoot[\pagemark]{\pagemark}
%\ohead[\headmark]{\headmark}
\onehalfspacing

\section{Code Smells (2P)}
\emph{[jeweils 1 Code-Beispiel zu 2 unterschiedlichen Code Smells aus der Vorlesung; jeweils Code-Beispiel
und einen möglichen Lösungsweg bzw. den genommen Lösungsweg beschreiben (inkl. (Pseudo-)Code)]}

\subsubsection{Long Method}
\noindent Dieser Code Smell war in der \enquote{calculateTurn}-Methode in der \enquote{ControlMechanismFour}-Klasse enthalten. Wie in \hyperref[fig:longmethod]{Abbildung 7.1} ersichtlich, ist die Methode viel zu lang. Zusätzlich mussten noch Kommentare eingefügt werden, dass die Methode überhaupt einigermaßen verständlich ist. 

Nach der Eliminierung des Code Smells sind nur noch die drei logischen Möglichkeiten nach den Spielregeln in der Methode enthalten. Das eigentliche Berechnen der Züge ist ausgelagert. Durch die Methodennamen sind auch keine Kommentare mehr notwendig.

\begin{figure}[htbp]
\centering
\centerline{\includegraphics[scale=.5]{longmethodsolution}}
\caption{Eliminierung der Long Method [Eigene Darstellung aus \emph{IntelliJ}]}
\label{fig:longmethodsolution}
\end{figure}

\begin{figure}[htbp]
\centering
\centerline{\includegraphics[scale=.65]{longmethod}}
\caption{Long Method [Eigene Darstellung aus \emph{IntelliJ}]}
\label{fig:longmethod}
\end{figure}

\newpage

\subsubsection{Switch-Statements}

\begin{figure}[htbp]
\centering
\centerline{\includegraphics[scale=.55]{graphutilities}}
\caption{Switch-Statements [Eigene Darstellung aus \emph{IntelliJ}]}
\label{fig:graphutilities}
\end{figure}

\noindent Der Code-Smell bezieht sich auf die vier Methoden der \enquote{GraphUtilities}-Klasse (\hyperref[fig:graphutilities]{siehe Abbildung 7.3}). Der Switch-Statements Code Smell gilt auch hier, da die if-else-Statements einfach durch switch-Statements ersetzt werden können. Da sich diese Methoden alle um das \enquote{FieldType}-Enum drehen, wäre an dieser Stelle eine Implementierung der Methoden im Enum sinnvoller und würden den Code-Smell entfernen. 

\newpage
\noindent Das Enum wurde um fünf Instanzvariablen erweitert:

\begin{figure}[htbp]
\centering
\centerline{\includegraphics[scale=.55]{instanzvariablen}}
\caption{Erweiterung um fünf Instanzvariablen [Eigene Darstellung aus \emph{IntelliJ}]}
\label{fig:instanzvariablen}
\end{figure}

\noindent Außerdem wurden die vier Funktionalitäten aus den Graph-Utilities hinzugefügt:

\begin{figure}[htbp]
\centering
\centerline{\includegraphics[scale=.8]{functionalities}}
\caption{Hinzufügen der vier Funktionalitäten [Eigene Darstellung aus \emph{IntelliJ}]}
\label{fig:funtionalities}
\end{figure}

\newpage
\section{2 Refactorings (6P)}
\emph{[2 unterschiedliche Refactorings aus der Vorlesung anwenden, begründen, sowie UML vorher/nachher
liefern; jeweils auf die Commits verweisen]}

\subsubsection{Extract Method}
\noindent Die Methode \enquote{saveGame} in der Klasse \enquote{GameIO} hat zuerst einen JSON-String erstellt und dann diesen mit dem aktuellen Zeitpunkt im Dateinamen abgespeichert. Beim Refactoring wurde hier einmal das Erstellen des JSON-Strings und das Generieren des aktuellen Zeitpunkts ausgelagert. Jetzt befindet sich nur noch der tatsächliche Abspeicherungsprozess in der Methode.

\begin{figure}[htbp]
\centering
\centerline{\includegraphics[scale=.5]{gameio1}}
\caption{Extract Method (vorher) [Eigene Darstellung aus \emph{IntelliJ}]}
\label{fig:gameio1}
\end{figure}

\newpage
\titlespacing*{\chapter}{0pt}{-30mm}{10pt}
\titleformat{\chapter}[display]
  {\normalfont\bfseries}{}{10pt}{\Huge\thechapter.\quad}
  
\chapter{Entwurfsmuster (8P)}
\pagestyle{scrheadings}
\clearscrheadfoot
\pagenumbering{arabic}
\setcounter{page}{8}
\ofoot[\pagemark]{\pagemark}
%\ohead[\headmark]{\headmark}
\onehalfspacing

\emph{[2 unterschiedliche Entwurfsmuster aus der Vorlesung (oder nach Absprache auch andere) jeweils
sinnvoll einsetzen, begründen und UML-Diagramm]}

\subsubsection{Entwurfsmuster: [Name] (4P)}
\subsubsection{Entwurfsmuster: [Name] (4P)}

\newpage
\titlespacing*{\chapter}{0pt}{-30mm}{10pt}
\titleformat{\chapter}[display]
  {\normalfont\bfseries}{}{10pt}{\Huge\thechapter.\quad}
  
% \chapter{Fragen an M. Müller}
% \pagestyle{scrheadings}
% \clearscrheadfoot
% \pagenumbering{arabic}
% \setcounter{page}{9}
% \ofoot[\pagemark]{\pagemark}
% %\ohead[\headmark]{\headmark}
% \onehalfspacing

% Fragen: 

% \begin{itemize}
%     %\item Dürfen wir \enquote{wir} in diesem Dokument verwenden?
%     %\item Wie genau soll die eigene Erklärung der Clean Architecture sein?
%     \item Siehe Kapitel SOLID / SRP: Ist es ok, dass wir die Klasse ControlMechanismFour nach refactoring nicht mehr haben?
%     \item Kapitel 3.2 OCP: Formulierung \enquote{Problem}
%     \item Kapitel 4.1 GRASP: Meint er mit negativ Beispiel einer geringen Kopplung eine starke Kopplung? 
%     \item Und soll die Kopplung in beiden fällen komplett aufgelöst werden oder nur zu einer geringen Kopplung gemacht werden?
%     \item Verstehen wir das richtig, dass man im DDD nur ein Repository haben kann, wenn man einen persistenten Speicher hat?
% \end{itemize}
%\include{kapitel6}
%\include{kapitel7}
% \renewcommand{\bibname}{}
% \clearscrheadfoot
% \pagenumbering{Roman}
% \setcounter{page}{9}
% \ofoot[\pagemark]{\pagemark}
% \vspace*{2.4cm}
% \huge\textbf{Literaturverzeichnis}
% \phantomsection
% \addcontentsline{toc}{chapter}{Literaturverzeichnis}
% \vspace*{.7cm}
% \printbibliography[heading=none]

% Ab hier beginnt der Anhang
%\appendix
%\addcontentsline{toc}{chapter}{Anhang}
%
%\addcontentsline{toc}{chapter}{Index}
%\printindex
%
%\addcontentsline{toc}{chapter}{Literaturverzeichnis}

% Hat man das "biblatex"-Paket nicht installiert, benutzt man folgendes:
% Ohne das "biblatex"-Paket (s. bericht.sty) produziert folgendes
% "deutsche" Zitate in Literaturverzeichnissen gemaß der Norm DIN 1505,
% Teil 2 vom Jan. 1984.
% Die Zitatmarken werden alphabetisch nach Verfassern
% sortiert und sind durch abgekürzte Verfasserbuchstaben plus
% Erscheinungsjahr in eckigen Klammern gekennzeichnet.

% \bibliographystyle{alphadin}
% \bibliography{bericht}

%%%%%%%%%%%%%%%%%%%%%%%%%%%%%%%%%%%%%%%5
% BIBLATEX
% Benutzt man das "biblatex"-Paket, muss man folgendes schreiben:

%%%%%%%%%%%%%%%%%%%%%%%%%%%%%%%%%%%%%%%5


%\include{changelog}

%\newpage
%\addcontentsline{toc}{chapter}{Liste der ToDo's}
%\listoftodos[Liste der ToDo's]


\end{document}
